\documentclass{article}
\usepackage[utf8]{inputenc}

\usepackage[margin=2.5cm]{geometry}
\usepackage[parfill]{parskip}

\usepackage{float}
\usepackage{tcolorbox}

\title{Total Turkey Donkey\\\large{Interaktiv 2D plattformare}}
\author{Christoffer Bohman \\ Sebastian Lorensson
}
\date{\today}

\begin{document}

\maketitle

%%%%%%%%%%%%%%%%%%%%%%%%%%%%%%%%%%%%%%%%%%%%%%%%%%%%%%%%%%%%%%%%%%%%%%%%%%%%%%%%%%%%
%%%%%%%%%%%%%%%%%%%%%%%%%%%%%%%%%%%%%% SPELIDÉ %%%%%%%%%%%%%%%%%%%%%%%%%%%%%%%%%%%%%
%%%%%%%%%%%%%%%%%%%%%%%%%%%%%%%%%%%%%%%%%%%%%%%%%%%%%%%%%%%%%%%%%%%%%%%%%%%%%%%%%%%%

\section{Spelidé}

Applikationen är ett 2D plattformsspel starkt inspirerat av \textbf{Ultimate Chicken Horse}. Spelet är ett multiplayer spel och går ut på att navigera en bana och nå en slutzon för att få ett poäng. Spelet delas in i  1 - n st rundor per match. Spelaren med flest poäng i slutet av en match vinner.

När applikationen börjar kan upp till 4 spelare välja att vara med och spela. Spelaren benämnd \textbf{Player 1} får sedan välja spel-läge: \textbf{hot-seat} eller \textbf{simultaneous}. De kan också välja hur många rundor som ska spelas, hur lång (om någon) tidsbegränsning rundorna ska ha och eventuellt fler.

\subsection{Spel-loop}
En runda är indelad i två delar: en nivå-redigerings-del och en plattformsspelar-del.
\begin{enumerate}
    \item I börjar av en runda får alla spelare tillgång till level-editorn och får sätta ner ett objekt i spelvärlden. När alla spelare har valt ett objekt börjar real-tids-delen av rundan.
    \item Baserat på inställningarna spelar nu rundan ut på ett av två sätt:
    \begin{enumerate}
        \item \textbf{Simultaneous (sv. Samtida):} Alla spelare spelar banan samtidigt och den första som når slutzonen får ett poäng.
        \item \textbf{Hot-seat (sv. Het potatis):} Spelare turas om att spela banan, fr.o.m. \textbf{Player 1} upp till \textbf{Player 4}, baserat på hur många som valdes i huvudmenyn. Det spelar på samma sätt som det samtida läget, men varje spelare kan få varsitt poäng.
    \end{enumerate}
    \item Gå tillbaka till 1. eller avsluta matchen om antalet rundor spelade är samma som antalet som valdes i huvudmenyn.
\end{enumerate}

\subsection{Spelidé forts.}
Komplexiteten i spelet uppstår från faktumet att de föremål som spelare placerar i början på varje runda kvarstår under spelets gång, vilket skapar ett behov till att balansera hur svår banan är för motståndarna och för en själv.

Exempel på objekt som spelare kan sätta ner:
\begin{itemize}
    \item armborst som skjuter pilar i en rät linje längs armborstets riktning
    \item trampoliner som skjuter iväg spelarens karaktär
    \item rörliga plattformar som spelare kan åka på
    \item eldtunnor som dödar spelaren karaktär om de landar på dem
    \item hammare som förstör placerade objekt i spelnivån
\end{itemize}

En spelare förlorar en runda på ett av tre sätt:

\begin{enumerate}
    \item Spelarens karaktär dör, antingen av:
    \begin{enumerate}
        \item Spelarens karaktär faller ur banan
        \item Spelarens karaktär tar skada från ett föremål
    \end{enumerate}
    \item Tiden tar slut
    \item En annan spelare når flaggan först (endast spelläge \textbf{(a))}.
\end{enumerate}

\section{Redigeringsverktyg}

Spelets banor är designade för att användas med ett redigeringsverktyg. Varje bana laddas från en fil och hålls i minnet under spelets gång. I början på en runda \textbf{1.} får varje spelare välja från ett urval av 3 slumpade föremål som de sedan kan placera i spelbanan genom att translatera och rotera föremål längs med ett rutnät. 

Om det finns tid kommer spelare även kunna göra egna banor och spara dem, med ett urval av vilka föremål som kan dyka upp (föremåls-poolen är slumpad men bara de valda föremålen är i slump-poolen).
\end{document}
